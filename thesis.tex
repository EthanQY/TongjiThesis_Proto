\documentclass[degree=master, degreetype=profession, bibtype=numeric, newstyle=true]{tongjithesis}
% 选项:
%   degree=[master|doctor],                             % 必选
%   bibtype=[numeric|authoryear],                       % 可选, 数字式引用 | 作者-年份引用, 默认为数字式引用
%   degreetype=[academic|profession|equaleducation],    % 可选, 学术型 | 专业型 | 同等学力, 默认为学术型
% 	electronic,                                         % 可选, 电子版标注: " 打印时删除 "
%   secret,                                             % 可选, 是否保密: 秘密 / 机密 / 绝密
%   pifootnote,                                         % 可选, 脚注是否用 ① 这种格式, 默认已打开
%   newstyle,                                           % 可选, 是否使用新页眉格式 (居中格式), 默认已打开
%   注:默认已打开的选项可以使用arialtitle=false 的形式关闭

% 其他文本格式 / 章节标题格式等选项可以参见 tongjithesis.cfg 中的说明, 按需要修改

%%%%%%%%%%%%%%%%%%%%% 导言区 %%%%%%%%%%%%%%%%%%%%%
% 环境准备
\wyqymathversion{9}                                     % 填写本文件中 wyqyinputfile 命令的使用次数; 可以比实际数量多, 但不可少
\usepackage{tongjiutils}                                % 自定义宏放在这里, 包括一些常用指令
\addbibresource{ref/refs.bib}                           % 加入bib文件

%%%%%%%%%%%%%%%%%%%%% 正文区 %%%%%%%%%%%%%%%%%%%%%
\begin{document}

% 定义图片文件存放位置( 默认在 figures 子目录下 )
% 最好使用 pdf 格式的图片
\graphicspath{{figures/}}

%%% 封面部分
\frontmatter
\wyqyinputfile{data/cover} \makecover                   % 封面示例, 位置: data/cover.tex
\tableofcontents                                        % 目录( 自动生成 )
\wyqyinputfile[denotation]{data/denotation}             % 符号对照表示例, 位置: data/denotation.tex
%%% 以下索引按需要选择, 不需要的注释掉即可
\listoffigures                                          % 插图索引
\listoftables                                           % 表格索引
\listofalgorithms                                       % 算法索引
% \listofequations                                      % 公式索引

%%% 正文部分
\mainmatter
% 用如下格式的命令插入文件, 格式: \wyqyinputfile[环境(可选参数)]{文件的相对路径}
\wyqyinputfile{data/chap01}                             % 第一章示例, 位置: data/chap01.tex
\wyqyinputfile{data/chap02}                             % 第二章示例, 位置: data/chap02.tex

%%% 封底部分
\backmatter
\wyqyinputfile[acknowledgement]{data/ack}               % 致谢示例, 位置: data/ack.tex
\wyqyinputfile[reference]{data/reference}               % 参考文献示例, 位置: data/ref.tex  
\wyqyinputfile[appendix]{data/appendix}                 % 附录示例, 位置: data/appendix.tex
\wyqyinputfile[resume]{data/resume}                     % 个人简历示例, 位置: data/resume.tex
\wyqyinputfile[statements]{data/statements}             % 原创性声明和授权书( 仅供测试, 最终提交时请替换为签字扫描件 )

\end{document}
