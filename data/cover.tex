\tongjisetup{
  %******************************
  % 注意:
  %   1. 配置里面不要出现空行
  %   2. 不需要的配置信息可以删除
  %******************************
  % 秘级
  secretlevel={公开},
  secretyear={0},
  %
  %
  % 中文信息
  ctitle={中文论文标题, \\ 中文论文副标题}, % 可手动加入换行符 \\ 控制换行
  cauthor={姓名},
  studentnumber={1234567},
  cdepartment={电子与信息工程学院},
  csubjectcategory={工学}, % 学科门类
  cmajorfirst={一级学科 \& 专业学位类别}, % 专业学位类别
  cmajorsecond={二级学科 \& 专业领域}, % 专业领域
  cresearchfield={研究方向}, % 研究方向
  csupervisor={导师姓名 教授}, 
  cassosupervisor={副导师姓名 教授}, % 副指导老师/行业导师, 没有把{}中内容留空即可, 或者注释掉
  cjointinstitution={联合培养单位名称}, % 联合培养单位, 没有把{}中内容留空即可, 或者注释掉
  cdate={\zhdigits{2024}年\zhnumber{3}月}, % 手动指定日期, 否则自动使用当前时间
  cfunds={(本论文由国家杰出青年基金 (No.123456789) 支持)}, % 中文版基金信息
  % 英文信息
  etitle={Thesis Title in English}, 
  eauthor={Nome Cognome},
  edepartment={College of Electronic and Information Engineering},
  esubjectcategory={Engineering}, % 学科门类
  emajorfirst={First-level Discipline \& Degree}, % 专业学位类别
  emajorsecond={Second-level Discipline \& Degree's Field}, % 专业领域
  eresearchfield={Research Fieldd}, % 研究方向
  esupervisor={Prof. Supervisor},
  eassosupervisor={Prof. Associate Supervisor}, % 副指导老师/行业导师
  ejointinstitution={Joint Training Institution}, % 联合培养单位
  edate={March, 2024}, % 手动指定日期, 否则自动使用当前时间
  efunds={(Supported by the Natural Science Foundation of China for Distinguished Young Scholars, Grant No.123456789)} % 英文版基金信息
}

\makeatletter
% 设置需要后处理的信息
\define@key{tongji}{cheadingtitlenew}{\cheadingtitlenew{#1}}
\gdef\cheadingtitlenew#1{\gdef\tongji@cheadingtitlenew{#1}}
% 后处理: 在标题中含有 \\ 的情况下, 删除 \\
\StrSubstitute{\tongji@ctitle}{ \\ }{ }[\tongji@post@title@1]
\StrSubstitute{\tongji@post@title@1}{ \\}{ }[\tongji@post@title@2]
\StrSubstitute{\tongji@post@title@2}{\\ }{ }[\tongji@post@title@3]
\StrSubstitute{\tongji@post@title@3}{\\}{ }[\tongji@post@title@4]
\cheadingtitlenew{\tongji@post@title@4}
\makeatother

% 定义中英文摘要和关键字
\begin{cabstract}
  \LaTeX 是常用的排版软件之一,能够输出高质量的学术出版物。
  \tongjithesis{} 是基于 \LaTeX 的模板,历经多次改进。本次改进的主要贡献如下:
  
  \begin{enumerate}[1.]
    \item 将平台迁移到 \emph{vscode};
    \item 适配最新的同济大学硕士学位论文要求;
    \item 添加了包括 \emph{tikz}、\emph{algorithm2e} 等宏包的支持。
  \end{enumerate}

  感谢使用!

\end{cabstract}

\ckeywords{LaTeX, 毕业论文, 模板, 同济大学}

\begin{eabstract}
  \LaTeX is the most popular typesetting system for academic writing.
  \tongjithesis{} is a \LaTeX template for Tongji University. The main contributions of this modification are as follows:

  \begin{enumerate}[1.]
    \item Migrate the platform to \emph{vscode};
    \item Adapt to the latest requirements of the master's thesis of Tongji University;
    \item Add support for packages such as \emph{tikz} and \emph{algorithm2e}.
  \end{enumerate}

  Thanks for using!
\end{eabstract}

\ekeywords{LaTeX, thesis, template, tongji-university}
